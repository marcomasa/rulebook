% !TEX root = ../Rulebook.tex

In the medium term, the \RCAW aims to transfer specific aspects of industrial scenarios in the tests and to demonstrate the practical applicability of the solutions. The challenges, which are adapted or redefined annually, serve as a test platform for the further development of the competitions. Each technical challenge is separately awarded. That means, teams can participate in any number of them.

A challenge increases the level capabilities of a robot in \RCAW related to:

\begin{itemize}
  \item \textbf{Variability of the environmental conditions} ... The setup conditions of a run are designed variably including disturbances. The lighting situations in the arena are changed dynamically, the configuration of the tables (height, format) is adapted or manipulation objects are mixed with unknown decoy objects.
  \item \textbf{Complexity of the scenarios} ... New arena elements are involved in a scenario
  or its dimensions (size, duration) are increased. This includes, for example, multi-robot scenarios, assembly tasks or new interaction stations.
\end{itemize}

For a successful implementation either an existing solution has to be increased in robustness or a new approach for an additional task has to be developed. The challenges here lie in the fields of perception, manipulation, navigation and planning.

\begin{figure}[h!]
  \centering
  \begin{tikzpicture}
  	\begin{polaraxis}[
        xtick ={0, 90, 180, 270},
        xticklabels= {Manipulation, Perception, Navigation, Planning},
        width=5.5cm,
        height=5.5cm,
        legend pos=outer north east,
    ]
  	\addplot
  		coordinates {(0,1.5) (90,3) (180, 0.5) (270,0.5) (0, 1.5)};  % cp
      \addlegendentry{Cluttered Pick Test}
    \addplot
    	coordinates {(0,3) (90,1.5) (180,1) (270,1) (0,3)};  % pfd
      \addlegendentry{Pick from Drawer Test}
  	\end{polaraxis}
  \end{tikzpicture}
  \label{Examplary Challenge introducing a long term operation based on an extended Final test}
\end{figure}

The challenges of 2020 focussing on perception and manipuation two scenarios. While "Cluttered Pick Test" (CP) adresses the robustness of perception, the "Pick from Drawer Test" (PFD) is focused on additional complexity by including objects in a drawer.


\newpage
\section{Cluttered Pick Test}

\subsection{Purpose and Focus of the Test}

The purpose of the \iaterm{Cluttered Pick Test}{CPT} is to evaluate the perception
and manipulation of the robots when objects are not seperated.

The scenario is motivated by the fact that most of the objects in the factory or 
labs are not perfectly placed but may be stacked or cluttered across a table.

Robots should be able to successfully grasp such objects in a way that they can place them somewhere else.
This ensures that they pick objects regarding potential further use.

\subsection{Scenario Environment}

The scenario is an alternated Basic Manipulation Test, with two service areas with a height of 10cm involved.
All available objects must be placed randomly in a box, which then is emptied above one table. The positions of the objects must remain as they fall. Objects that end up outside of the manipulation zone may be gathered, placed in the box and dropped again. The rule for a minimum distance of 0.02m between objects does not apply for this test. They may be placed near and on top of each other (see \ref{fig:clutter}).

\begin{figure} [h!]
\begin{center}
\subfloat[]{\includegraphics[height = 3cm]{./images/cluttered_pick_1.jpg}} \hspace{1cm}
%\subfloat[]{\includegraphics[height = 3cm]{./images/cluttered_pick_2.jpg}}
\end{center}
\caption{objects places in cluttered environment}
\label{fig:clutter}
\end{figure} 


%\subsection{Variation}
%A slight variation in the competition will be to run with 2 robots simultaneously competing with each other.
%For fairness of the competition the organizers should ensure that the distance between the robot starting
%point and the corresponding service station be equal.

\subsection{Task}

The task is the same as for a BMT, with the modification that only 3 objects must be picked and placed.

\subsection{Rules}
The following rules have to be obeyed:

\begin{itemize}
\item A single robot is used.
\item Three objects have to be picked.
\item There must be atleast 5 decoy objects which must not be picked.
\item The robot has to start from outside the arena and to end in the final.
\item A manipulation object counts as successfully grasped as specified in Section~\ref{ssec:GraspingObjects}.
\item The run is over when the robot reached the final place or the designated time has expired.
\item The order in which the teams have to perform will be determined by a draw.
\item At the beginning of a team's period, the team will get the task specification.
\end{itemize}

\subsection{Scoring}
\begin{itemize}
\item 200 points are awarded for each correctly and successfully picked object
\item 125 points are awarded for each correctly and successfully placed object
\item -100 points for every incorrectly picked object
\end{itemize}

\newpage
\section{Pick from Drawer Test}

%\szug{Should we think about additional rules for collisions?}

\szug{I think an individual drawer avoids unpredictable configurations but generates of course additional effort. Probably we think about a "standard drawer construction" for 2021?}

%\szug{Should we include one or two decoy objects that are placed by a referee to avoid scripted solutions.}

\subsection{Purpose and Focus of the Test}
The collection of freely available objects lying on a manipulation zone is the core capability of \RCAW-robots. The \iaterm{Pick from Drawer Test}{PFD} goes beyond this level and considers objects stored in drawers too. In this way, the challenge
extends the idea of the shelf where the robot has to plan the grasping operation in a limited space but it is not necessary to interact with the environment.

\subsection{Scenario Environment}
The first version of the challenge gives much freedom to the teams. They can choose an arbitrary drawer configuration. The drawer is wholly covered in the beginning and can only be linearly moved in one direction parallel to the floor. The inside of the drawer has to have a uniform color and an uniform flat surface. The surrounding construction has to remain at its position. The inside and the handle of the drawer count as manipulation zones for collision detection purposes.

\par
Robot's movements must open the drawer directly. Self-driven, automatic solutions integrated into the drawer system are not allowed.
The rules do not define the handling mechanism itself, the teams are completely free to design an appropriate concept. Any handle, knob, hole, or connector mounted to the drawer is permitted. Based on this interaction, the drawer has to be moved at least \textcolor{red}{15 cm}. 

\subsection{Task}
The drawer setup is located at an arbitrary position. The drawer contains randoly chosen manipulation objects to be picked described in Table~\ref{tab:Instances}. It is stored directly on bottom of the drawer. Additionally, the drawer may contain three decoy objects and an arbitrary surface. 

The team configures the objects and the drawer during preparation phase.

\textcolor{red}{This test does not adress navigation capabilities. Hence the robot can start the run anywhere in the arena.} It moves directly to the drawer, opens it, grasps the objects and place them in the object inventory of the robot.

\subsection{Rules}
The following rules have to be obeyed:

\begin{itemize}
\item A single robot is used.
\item The test runs for 5 minutes.
\item \textcolor{red}{The robot can start at an arbitrary position inside the arena.}
\item The order in which the teams have to perform will be determined by a draw.
\item Each team is responsible for preparing the drawer system. The team selects the objects and places them within the drawer.
\item The drawer is opened by at least \textcolor{red}{15cm}.
\item A manipulation object counts as successfully grasped as specified in Section~\ref{ssec:GraspingObjects}. \textcolor{red}{It is not necessary to place the objects at another manipulation zone.}
\item The run is over when the designated time has expired.
\end{itemize}

\subsection{Scoring}
\begin{itemize}
\item 100 points for opening the drawer
\item 100 points are awarded for each correctly and successfully picked object, +50 Points per object if decoys are present, +50 Points per object if the Abritrary Surface is present.
\item Time bonus of one point per second after collecting 3 objects successfully.
\end{itemize}

\newpage
\section{Simulation Evaluation Test}

\subsection{Purpose and Focus of the Test}

The purpose of this test is to provide the RoboCup @Work League with new capabilities. These capabilities are the option to do scientific evaluation regarding stochastic behaviour and scalability analysis. This provide the competing teams with the option of using their experimental results in scientific papers and provide a stronger link to the scientific robotic communities.

Another aspect is the option to add integration tests and continuous integration to the workflow of the teams to provide better management of software versions. Additionally, this provides the team members with the capabilities to learn state-of-the-art software development techniques.

Finally, a simulation adds the option for new teams to start with a virtual robot excluding the typical hardware problems associated with real robots. This eases the entry into the league and paves the way for a larger growth of the league in regard to participating teams in the future. 

\subsection{Scenario Environment}

The scenario for this test is to enable teams using a (partial) simulation to show these to the league. Finally, the league may be able to choose a default simulation environment to provide support for this environment in the future. 

Consequently, the simulation of the team competing in this challenge needs to fulfill some requirements:

\begin{description}
  \item[Free to be used:] The simulation software needs to be usable by competing teams free of charge. The software does not need to be open source.
  \item[Open-Source API:] The interface of the simulation needs to be open source. Especially, the implementation of the robot specific functions and behaviours, like executing movement commandos and outputting laser scanner data etc.\ needs to be implemented in a way that allows for easy modification of interested teams.
  \item[Official Models:] The simulated arena environment need to contain the 3D-Model of the official repository of the @Work League \url{https://github.com/robocup-at-work/models}. Additionally, the tasks to be executed need to be generated by the official Referee Box, see Section~\ref{sec:refbox}
\end{description}

  Within the simulation environment one of the tasks specified in Section~\ref{sec:tests} needs to be executed.

\subsection{Task}

The task of this challenge is to show the execution of one of the tasks defined in Section~\ref{sec:tests} in the virtual environment. However, this task is not graded regarding the normal scoring scheme. The evaluation of this task is based on the behaviour of the simulation itself. Relevant aspects that are considered in the scoring are the precision and speed of the simulation. To this end, the teams shall provide stochastic data on the precision of multiple runs of their simulation  as well as the speed of the simulation expressed as a real-time-factor (Quotient between time passed in the simulation and time passed in the real world). Additionally, the teams need to indicate the API of their simulation as well as the used simulation software and its license. The task execution may either be shown in a video or live. 

\subsection{Rules}

\begin{itemize}
  \item Virtual representation based on the object and table definitions from \url{https://github.com/robocup-at-work/models}
  \item Virtual representation of the teams robot
  \item Free to use (for robotic teams) simulation software / environment
  \item Execution of a task as specified in Section~\ref{sec:tests}
  \item Start of task by Referee Box see Section~\ref{sec:refbox}
  \item Task execution as video or live
  \item Indication of precision in form of reproduction accuracy (execute multiple times and compare results)
  \item Indication of simulation speed based on real-time factor (Quotient between virtual clock speed and real world time)
\end{itemize}

\subsection{Scoring}

Referees grade task execution and simulation based on the following criteria:
\begin{itemize}
  \item Up to 100 Points for Ease of Use
  \item Up to 100 Points for Visualization
  \item Up to 200 Points for Precision
  \item Up to 200 Points for Speed
  \item Up to 300 Points for Simulation Capabilities
\end{itemize}



