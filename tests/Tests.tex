% !TEX root = ../Rulebook.tex
\label{sec:tests}

The actual competition contains of a set of so-called tests. 
A test is specified in terms of it's purpose and focus, environment features and eventually manipulation objects involved. Further, a concrete specification of the task is given and the rules to be obeyed. 

Each test has different variability dimensions. That is, which objects to be manipulated, how many locations to visit, from which height to grasp etc. The test instances for \YEAR are defined based on the general test description and can be seen in Section~\ref{sec:ScoringAndRanking}.

\input{./tests/BNT}

\input{./tests/BMT}

\input{./tests/BTT}

\input{./tests/PPT}

\input{./tests/RTT}

%\newpage
%\section{Test Variability} \label{sec:TestVariability}

%The different optional parameters and configurations for each task are 
%mentioned in Section~\ref{sec:ArenaDesign} and \ref{sec:ManipulationTasks}. 
%Figure~\ref{fig:complexityTree} summarizes the possible variations and 
%emphasizes aspects that may be chosen.

%\begin{figure}[ht]
%\centering
%\input{./tikz/complexityTree.tikz}
%\caption{Aspects of variability that may be integrated in a specific instance of a test.}
%\label{fig:complexityTree}
%\end{figure}
